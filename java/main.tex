\documentclass[a4paper,12pt]{article}
% packege for margine
\usepackage{geometry}
\newgeometry{
    top=0.7in,
    bottom=0.7in,
    outer=0.7in,
    inner=0.7in
}

\title{Superiority of Lagrangian}
\author{Debojyoti Tantra}
\date{May 2025}

\begin{document}

% \maketitle

\section*{\centering{\underline{\textbf{Superiority of Lagrangian over Newtonian Approch}}}}

    \quad In the Newtonian mechanics, the equations of motion involve vector quantities like force, momentum etc. which increase complexity in solving the problems. This approach also cannot avoid constraints presem in a problem. These forces of constraints, if not known, make the solution of the problem difficult and even if they are known, the use of rectangular or other commonly used coordinates may make the solution of the problem to be impossible. These drawbacks are removed in the Lagrangian mechanics, where the technique involves scalars, like potential and kinetic energies, instead of vectors. The use of generalized coordinates in the Lagrangian formulation often allows automatically for the constraints. In this formulation, the difficulty 2 in solving the problems is many times much reduced, when any quantity like momentum or (length) is taken as a generalized coordinate instead of rectangular or commonly used coordinates. Further the form of the Lagrange's equations of motion remains invariant under any generalized coordinate transformation.

    \subsection*{Lagrangian Equation}
        \begin{equation}
            \frac{d}{dt} \left( \frac{\partial L}{\partial \dot{q_j}} \right) -  \frac{\partial L}{\partial q_j} = 0
        \end{equation}

\end{document}
